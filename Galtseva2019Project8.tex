\documentclass[12pt,twoside]{article}
\usepackage{jmlda}
\usepackage[utf8]{inputenc}
\usepackage[russian]{babel}
\usepackage[T2A]{fontenc}
\usepackage{lineno}
\linenumbers
\usepackage{setspace}
\doublespacing

\usepackage[left=1.5cm,right=1.5cm,
    top=2cm,bottom=2cm,bindingoffset=0cm]{geometry}

\title
    [Порождение признаков с помощью локально-аппроксимирующих моделей]
    {Порождение признаков с помощью локально-аппроксимирующих моделей}
\author {ГАЛЬЦЕВА А. И.} % основной список авторов, выводимый в оглавление
\email{alex.galtseva@gmail.com}
\thanks
    {Научный руководитель:  Стрижов~В.\,В.}
\organization
    {$^1$Московский физико-технический институт (МФТИ)}
\abstract
    { В данной статье рассматривается проблема определения вида деятельности (отдых, работа) рабочего по данным акселерометра и гироскопа, для решения которой реализованы несколько подходов порождения и классификации признаков -- оптимальных параметров локально-аппроксимирующих моделей для сегментов исходных временных рядов. Статья посвящена исследованию свойств параметров локально-аппроксимирующих моделей - признаками сегментов временных рядов. 

\bigskip
\textbf{Ключевые слова}: \emph {классификация временных рядов, линейно-аппроксимирующие модели, выборка параметров}.}

\begin{document}
\maketitle



\section{Введение}
 Задача семантической интерпретации временных рядов - однa из самых распространенных в машинном обучении, которая рассмотрена в работах[]. Цель работы - исследовать свойства выборки признаков временных рядов - оптимальные параметры локально-аппроксимирующих моделей. Предлагается распознавать вид физической активности по временному ряду. Данная задача была решена в работах[].
 Достичь поставленой цели плануется выполнением следующего плана:
 1) Найти и предобработать данные.
 2) Сегментировать полученные ВР. Как описано в  \cite{motrenko}.
 3) Использовать локально-аппроксимирующе модели, рассмотренные в работах[1, 2]
 4) Исследовать свойства оптимальных параметров локально-аппроксимирующих моделей
 5) Реализовать классификацию сегментов ВР
 6) Сделать выводы о связи свойств оптимальных параметров локально-аппроксимирующих моделей  с результатами классификации 
 



%\bibliography{Galtseva2019Project8}{}
\bibliographystyle{plain}
\end{document}
